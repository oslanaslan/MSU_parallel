\documentclass[a4paper, 12pt]{article}

\usepackage{a4wide}
\usepackage[utf8]{inputenc}
\usepackage[russian]{babel}
\usepackage[pdftex]{graphicx}
\usepackage{amsmath}
\usepackage{amssymb}
\usepackage{amsfonts}
\usepackage{graphicx}
\usepackage{hyperref}
\usepackage{verbatim}
\usepackage{indentfirst}
\usepackage[left=3cm, right=1.5cm, top=2cm, bottom=2cm]{geometry}
\setlength{\parindent}{1.25cm}
\linespread{1.3}
\setcounter{page}{2}

\newtheorem{myComment}{Замечание}
\newtheorem{myDef}{Определение}
\newtheorem{myProp}{Свойство}
\newtheorem{myTeor}{Теорема}
\newcommand{\dbtilde}[1]{\accentset{\approx}{#1}}
\DeclareMathOperator{\sign}{sign}

\begin{document}

\thispagestyle{empty}

\begin{center}
\vspace{-3cm}
\includegraphics[width=0.5\textwidth]{sources/msu.eps}\\
{\scshape Московский государственный университет имени}\\
М. В. Ломоносова\\
Факультет вычислительной математики и кибернетики\\
Кафедра системного анализа

\vfill

% {\LARGE }

\vspace{1cm}

{\Huge\bfseries Суперкомпьютерное моделирование и технологии}\\

\vspace{1cm}

{\LARGE Задание 2}
\end{center}

\vspace{1cm}

\begin{flushright}
  \large
  \textit{Студент 615 группы}\\
  А.\,Н.~Ашабоков

  \vspace{5mm}

\end{flushright}

\vfill

\begin{center}
Москва, 2020
\end{center}

\newpage
\setcounter{tocdepth}{2}
\tableofcontents

\newpage
\normalsize

\section{Постановка задачи}

    Необходимо предоставить аналитическое решение и программную реализацию алгоритма численного решения задачи вычисления слудющего интеграла:
    
    \begin{equation}
        I = \int\int\limits_{G}\int \frac{dxdydz}{{(1 + x + y + z)}^3},
    \end{equation}
    где область $G$ ограничена поверхностями $x + y + z = 1$, $x = 0$, $y = 0$, $z = 0$.\\
    
    Также необходимо исследовать масштабируемость полученной программной реализации.

\section{Аналитическое решение}

\section{Описание численного алгоритма}

    Данная задача решается численно с использованием метода Монте-Карло: 
    \begin{enumerate}
        \item Задаем функцию $F(x, y, z)$ следующего вида:
            \begin{equation}
                F\left(x, y, z\right) = \left\{
                    \begin{array}{ll}
                        f\left(x, y, z\right) & \left(x, y, z\right) \in G, \\
                        0 & \left(x, y, z\right) \not\in G.
                    \end{array}
                \right.
            \end{equation}
        \item Далее преобразуем исходный интеграл:
            \begin{equation}
                I = \int\int\limits_{G}\int f\left(x, y, z\right)dxdydz = \int\int\limits_{G'}\int F\left(x, y, z\right) dxdydz.
            \end{equation}
            где $G'$ --- прямоугольник: $a_1 \leq x \leq b_1$, $a_2 \leq b_2$, $a_3 \leq b_3$.
        \item После этого семплируем случайные точки из $G'$ и на них считаем значение функции $F$.
        \item Окончательный результат получается из соотношения:
            \begin{equation}
                I' \approx \rvert G' \rvert \cdot \frac{1}{n} \sum\limits_{i = 1}^n F(p_i).
            \end{equation}
        \item Процесс продолжается до тех пор, пока значение ошибки не будет меньше некоторого наперед заданного значения $\varepsilon$: $|I - I'| < \varepsilon$.
    \end{enumerate}

\section{Результаты запуска программы}

\subsection{Bluegene}

    \begin{tabular}{|c|c|c|c|c|}
        \hline
        Точность $\varepsilon$ & Число MPI-процессов & Время работы (с) & Ускорение & Ошибка \\
        \hline
        1.0 \code 10^{-4} & 2 & 0.0152145408 & 1 & 0.0000664537\\
        1.0 \code 10^{-4} & 4 & 0.0145364892 & 1.05 & 0.0000894448\\
        1.0 \code 10^{-4} & 16 & 0.0003104247 & 49.01 & 0.0000379321\\
        1.0 \code 10^{-4} & 64 & 0.0001935188 & 78.62 & 0.0000136576\\
        \hline
        2.0 \code 10^{-5} & 2 & 0.0076551648 & 1 & 0.0000090474\\
        2.0 \code 10^{-5} & 4 & 0.0175335271 & 0.44 & 0.0000065640\\
        2.0 \code 10^{-5} & 16 & 0.0007019118 & 10.91 & 0.0000196344\\
        2.0 \code 10^{-5} & 64 & 0.0001932365 & 39.62 & 0.0000136576\\
        \hline
        8.0 \code 10^{-6} & 2 & 0.0104568517 & 1 & 0.0000056839\\
        8.0 \code 10^{-6} & 4 & 0.0175511012 & 0.59 & 0.0000065640\\
        8.0 \code 10^{-6} & 16 & 0.0007172153 & 14.58 & 0.0000024089\\
        8.0 \code 10^{-6} & 64 & 0.0002022035 & 51.71 & 0.0000021677\\
        \hline
    \end{tabular}

\subsection{Polus}

    \begin{tabular}{|c|c|c|c|c|}
        \hline
        Точность $\varepsilon$ & Число MPI-процессов & Время работы (с) & Ускорение & Ошибка \\
        \hline
        1.0 \code 10^{-4} & 2 & 0.0152145408 & 1 & 0.0000664537\\
        1.0 \code 10^{-4} & 4 & 0.0345364892 & 0.44 & 0.0000894448\\
        1.0 \code 10^{-4} & 16 & 0.0077306512 & 1.97 & 0.0000158393\\
        1.0 \code 10^{-4} & 64 & - & - & -\\
        \hline
        2.0 \code 10^{-5} & 2 & 0.0076551648 & 1 & 0.0000035639\\
        2.0 \code 10^{-5} & 4 & 0.0032367209 & 2.37 & 0.0000025931\\
        2.0 \code 10^{-5} & 16 & 0.0012922971 & 5.92 & 0.0000021561\\
        2.0 \code 10^{-5} & 64 & - & - & -\\
        \hline
        8.0 \code 10^{-6} & 2 & 0.0355211364 & 1 & 0.0000007309\\
        8.0 \code 10^{-6} & 4 & 0.0183235371 & 1.94 & 0.0000005321\\
        8.0 \code 10^{-6} & 16 & 0.0127631197 & 2.78 & 0.0000006193\\
        8.0 \code 10^{-6} & 64 & - & - & -\\
        \hline
    \end{tabular}

\newpage
\section{Библиография}

\begin{thebibliography}{}
	\bibitem{Kurzh} А.Б. Куржанский, П.А. Точилин К задаче синтеза управлений при неопределенности по данным финитных наблюдателей //
	 журнал Обыкновенные дифференциальные уравнения, том 47, номер 11, 2011г, сс.1599-1607 
	
\end{thebibliography}

\end{document}